\documentclass[a4paper,11pt,english]{book}
\usepackage[english]{babel}
\usepackage{listingsutf8,verbatim} %inclure du code 
\lstloadlanguages{R}
\usepackage[utf8]{inputenc} % Required for including letters with accents
\usepackage[T1]{fontenc} % Use 8-bit encoding that has 256 glyphs
\usepackage[dvipsnames]{xcolor}
\usepackage{xspace}
\usepackage{graphicx,epsfig,subfigure}
\usepackage{ragged2e}
\usepackage{fancyhdr}
\usepackage{booktabs}
\fancyhead[L]{\includegraphics [width=1cm]{images/Logo_ponts_paristech}}
\fancyhead[R]{\leftmark}
\usepackage[left=3.5cm,right=3.5cm,top=4.5cm,bottom=5.0cm]{geometry}  % Page margins
\usepackage{eso-pic}
\usepackage[Glenn]{fncychap}
\usepackage{array,supertabular,amsmath,amssymb,dsfont}
\usepackage{stmaryrd}
\usepackage{titlesec}
\usepackage{tablefootnote}
\usepackage{multicol,array}
\usepackage{float}
\usepackage{color}
\usepackage[most]{tcolorbox}
\usepackage{pgfplots}
\usepackage{xcolor}
\newcommand\BackgroundPic{%
\put(0,0){%
\parbox[b][\paperheight]{\paperwidth}{%
\vfill
\centering
\includegraphics[width=\paperwidth,height=\paperheight,%
keepaspectratio]{background.jpg}%
\vfill
}}}
\setlength\belowcaptionskip{0.3cm}
\setlength\abovecaptionskip{0.3cm}
\usepackage[hang,small]{caption}
\usepackage{multirow} 
\usepackage{multicol}
\usepackage{colortbl}
\usepackage{textcomp,eurosym} 
\setlength{\headheight}{36.06802pt}
\renewcommand{\footrulewidth}{0.4pt}
\renewcommand{\headrulewidth}{0.4pt}
\fancyhead[R]{\leftmark}
\renewcommand{\arraystretch}{1.2} %aérer tableaux 
\definecolor{dkgreen}{rgb}{0,0.6,0}
\definecolor{gray}{rgb}{0.5,0.5,0.5}
\definecolor{mauve}{rgb}{0.58,0,0.82}
\usepackage{stmaryrd} %double crochet
\usepackage{amsthm}
\usepackage{mathrsfs,amsmath} 
\newtheorem{prop}{Proposition}
\newtheorem*{theorem}{Théorème}
\usepackage[left=3.0cm,right=3.0cm,top=5.5cm,bottom=5.5cm]{geometry} % Page margins
\usepackage[colorlinks=true,urlcolor=blue,linkcolor=blue]{hyperref}
\let\cleardoublepage\clearpage
\definecolor{ballblue}{rgb}{0.13, 0.67, 0.8}
\definecolor{orange}{RGB}{242,239,121}
\definecolor{darkorange}{RGB}{229,186,27}
\linespread{1.05}
\newcommand{\Lim}[1]{\raisebox{0ex}{\scalebox{1}{$\displaystyle \lim_{#1}\;$}}}
\usepackage{longtable}
\usepackage{etoolbox,siunitx}
\usepackage{tabularx}
\newrobustcmd*{\bftabnum}{%
  \bfseries
  \sisetup{output-decimal-marker={\textmd{.}}}%
}
\usepackage{rotating}
\usepackage{collcell}

\newcommand{\setmaxnum}[1]{%
    \gdef\maxnum{#1}%
}
\newcommand{\numtest}[1]{%
    \ifdim#1pt > \maxnum pt
        $\mathbf{#1}$%
    \else
        $#1$%
    \fi%
}
\newcolumntype{E}{>{\collectcell\numtest}r<{\endcollectcell}}

\begin{document}
\sisetup{detect-weight=true,detect-inline-weight=math}

\frontmatter
%----------------------------------------------------------------------------------------
%	TITLE PAGE
%----------------------------------------------------------------------------------------
\begin{titlepage}
\begin{center}
\begin{figure}[H] 
  %\begin{minipage}[b]{0.5\linewidth}
    \centering
    \includegraphics[scale=0.2]{images/Logo_ponts_paristech.png} 
    \vspace{4ex}
  %\end{minipage}
\end{figure}
% Title
\rule{\linewidth}{0.5mm} \\[0.4cm]
{ \LARGE \bfseries Pricing callable BRCs with Reflected Backward Stochastic Differential Equations \\
}
\rule{\linewidth}{0.5mm} \\[1cm]


Caroline \textsc{OSTER}

\noindent

\vspace{4cm}

{\today}
\end{center}
\end{titlepage}
\newpage

\pagestyle{empty}
\tableofcontents
\listoffigures
\listoftables

%--------------------------------------------------------------------------
%	Intro
%--------------------------------------------------------------------------
\mainmatter
\chapter*{Introduction}
\addcontentsline{toc}{chapter}{Introduction}
\vspace*{-2cm}

STATS on BRCs in Switzerland
%--------------------------------------------------------------------------
%	Chapter 1 
%--------------------------------------------------------------------------

\pagestyle{fancy}

\chapter{Description of a callable barrier reverse convertible}

\section{Definition of a \textit{Standard Reverse Convertible}}
In order to understand what a \textit{Barrier Reverse Convertible} is, we will first give a brief definition of a \textit{Standard Reverse Convertible} : this kind of financial instrument is a bond-like security that pays a coupon that is generally very high compared to the risk-free interest rate. This bond is combined with a short put option, as can be seen in \ref{fig:RC-payoff}.

\begin{figure}[!h]
    \centering
    \includegraphics[scale=0.7]{images/RC.png}
    \caption{Payoff profile of a \textit{Reverse Convertible}. Source : \cite{lindauer2008pricing}}
    \label{fig:RC-payoff}
\end{figure}

The payment is made as follows : if the price of the underlying ends above the strike at maturity, the nominal is given back as the final payoff. In this case, the final payoff is called the cap. On the contrary, if the price of the underlying ends under the strike at maturity, the final payoff is the one of a short put. All the coupons are paid during the lifetime of the asset.\\

Let us denote $K$ the strike of the asset and $S_{T}$ the price of the underlying at maturity $T$. The final payoff $P$ can hence been expressed as follow :
$P=\text{cap}-(K-S_{T})_{+}$.\\

With this kind of security, the investor generally expects the market to move higher or at least to remain stable, so he doesn't lose his capital and makes benefits with high coupons. Moreover, a \textit{Reverse Convertible} can be beneficial if the volatility is expected to decrease or to stay low in the future, as the investor is mainly shorting volatility.

\section{Definition of a \textit{Barrier Reverse Convertible}}
\label{sec:BRC-definition}

The \textit{Barrier Reverse Convertible} is slightly different because it contains a barrier on the level of the underlying : if the underlying doesn't reach the barrier during the lifetime of the asset, the nominal is given back at the end, but in the case it is, the put is activated and the final payoff is the one of a short put. Of course, coupons are paid according to the terms of the contract. The main difference with the \textit{Reverse Convertible} is thus the path-dependency of the asset. \\

Figure \ref{fig:BRC-payoff} summarizes the payoff profile of a \textit{Barrier Reverse Convertible}:

\begin{figure}[!h]
    \centering
    \includegraphics[scale=0.65]{images/BRC.png}
    \caption{Payoff profile of a \textit{Barrier Reverse Convertible}. Source : \cite{lindauer2008pricing}}
    \label{fig:BRC-payoff}
\end{figure}

The barrier is always a knock-in option\footnote{See Appendix \ref{appendix:down-in-put} for a reminder on barrier options}, but it can be either 'up' or 'down'. If the barrier is 'up', it means that it is reached if the level of the underlying is above it. If the barrier is down, it is reached if the level of the underlying is below it.

\section{Multi \textit{Barrier Reverse Convertibles}}
\label{sec:multi-BRC-definition}
Multi \textit{Barrier Reverse Convertibles} are a type of \textit{Barrier Reverse Convertibles} that have several underlyings. This changes the definition of how the barrier is reached and if the final value is above the strike. In both aspects, several definition can be considered :

\begin{itemize}
    \item \textbf{"WorstOf" case}: in this case, the underlying chosen to be compared with the barrier (or the strike) is the one with the worst performance regarding the barrier (or the strike).
    \item \textbf{"BestOf" case}: in this case, the underlying chosen to be compared with the barrier (or the strike) is the one with the best performance regarding the barrier (or the strike).
    \item \textbf{"Basket" case}: in this case, a linear combination of the performances of the underlyings is chosen to be compared with the barrier (or the strike).
\end{itemize}

The definition of the underlying chosen to be compared with the barrier can be different from the one chosen to be compared with the strike, which allows 9 different types of multi BRCs.

\section{Adding the callability}
To the regular (multi) BRCs can be added the callability by the issuer. This means that the issuer can buy the contract at several dates and at a certain amount specified in the contract. The amount of this redemption is generally the nominal plus the on-going coupon.

%--------------------------------------------------------------------------
%	Chapter 2 
%--------------------------------------------------------------------------

\pagestyle{fancy}

\chapter{Pricing Methodology with Reflected Backward Stochastic Differential Equations}
\label{chap:pricing-methodology}
Since the financial instrument we want to price is a multi-dimensional asset, the best choice of pricing methodology seems to be a Monte Carlo method. The only problem with it is that the price between the begining and the maturity of the contract is unknown, and more particularly at the times of possible redemption by the issuer (called times of callability). But the price of the asset at those particular times has to be known (and compared with the amount of redemption precised in the contract) in order to determine whether the callability is activated or not by the issuer. A first and naive idea would be to use a new Monte Carlo at each time steps needed, but this would be computationally too long and too complex to be efficient.  \\
A better solution would be to proceed in a backward manner, begining with the price at maturity (which is known thanks to the explicit final payoff profile) and coming back to the price at $t=0$ step by step. With this approach, inspired by the Longstaff-Swartz algorithm \cite{schwartz2001valuing}, the price can be known at all the possible redemption times.\\
Hence, the modelisation chosen here is a forward-backward stochastic differential equation system. The forward component, which only represents the diffusion of the underlyings, will be resolved in a classical way whereas solving the backward component will allow us to compute the price, taking into account the coupons and the callability at each timestep.\\

The whole methodology will be described in this chapter. First, we will present the modelisation chosen to solve the problem considered, and justify the hypothesis made within this context. Then will be explained the forward equation resolution, the final payoff computation and the computation of the price in a backward manner.


\section{Presentation of the SDE system}
\label{sec:SDE-presentation}
As explained above, the objective is to compute the solution of  a forward-backward stochastic differential equation (FBSDE) which can be written as follow :
$$S_{t} = S_{0}+\int_{0}^{t}b(s,S_{s})ds +\int_{0}^{t} \sigma(s,S_{s})dW_{s}$$
$$Y_{t} = \Phi(\textbf{S}) + \int_{t}^{T}f(s,Y_{s},Z_{s})ds -\int_{t}^{T}Z_{s}dW_{s}$$
$\textbf{S}=(S_{t})_{t\geq0}$ is the forward component of the system. It represents the underlyings and its dimension is $d$. $(Y_{t})_{t\geq0}$ represents the value of a replicating portfolio at time t and is unidimensional.\\
The process $W$ is a d-dimensional brownian motion defined on a probability space $(\Omega,\mathcal{F},\mathbb{P})$ provided with its natural filtration $(\mathcal{F)}_{t}$ and generated by $W$. 
$T$ is the maturity of our asset.\\

The function $f$ is called the driver function. It should be uniformly Lipschitz, \textit{ie} there exist a constant $C$ such that :
$$\forall (y_{1},z_{1}),(y_{2},z_{2}), |f(t,y_{1},z_{1})-f(t,y_{2},z_{2})|\leq C(|y_{1}-y_{2}|+|z_{1}-z_{2}|)$$

Antonelli \cite{antonelli1993backward} proved the existence of such a solution using a fixed point theorem when the function $b$ does not depend on $(Z_{t})_{t\geq0}$ (which is our case) and under the condition $CT<1$ with $C$ the Lipschitz constant of $f$. In our case, $C$ is the risk free rate (see \ref{subsec:choice-of-f}); it is small enough so that this condition is almost always verified.\\

Resolving the backward equation thus consists in computing a pair of processes $(Y,Z)$. We will more especially focus on the process $(Y_{t})_{t\geq0}$ which gives the price of the option.

\section{Resolution of the forward equation}
\subsection{Model and scheme of discretization for the underlyings}
\label{subsec:underlying-discretization}
The model used for the diffusion of the underlyings is the Black-Scholes model. As the pricing is done under risk-neutral probability, we have for each underlying $(S^{i}_{t})_{t\geq0}$:
$$
\begin{cases}
dS_{t}^{i}=r\text{d}t+\sigma_{i}\text{d}W_{t}^{i} \\
S_{0}^{i}=s_{0}^{i} 
\end{cases}
$$
where $r$ is the risk free rate.
If we take a time grid $(t_{0},t_{1},\ldots,t_{m})$, we know that the solution of that stochastic equation can be discretized as follows :
$$\begin{cases}
\forall k=1,\ldots,m,  S_{t_{k}}^{i}=S_{t_{k-1}}^{i}e^{(r-\frac{\sigma_{i}^{2}}{2})(t_{k}-t_{k-1})+\sigma_{i}W_{t_{k}}^{i}}\\
S_{t_{0}}^{i}=s_{0}^{i} 
\end{cases}
$$
with $W_{t_{k}}^{i} \sim \mathcal{N}(0,t_{k}-t_{k-1})$.\\
This is the discretization we will use for the diffusion of the underlyings.
\subsection{Taking into account the correlation}
Moreover, the brownian motions are all correlated : $$\forall (i,j) \in \{1,\ldots,d\}\times\{1,\ldots,d\},~ \mathbb{E}(\text{d}W_{t}^{i}\text{d}W_{t}^{j})=\rho_{ij}\text{d}t$$
In order to take into account these correlations, we should generate the brownians with a multivariate gaussian distribution with correlation matrix $\Sigma=(\rho_{i,j})_{(i,j) \in \{1,\ldots,d\}\times\{1,\ldots,d\}}$.\\

The first step is to generate $d$ independent realizations of a standard normal distribution in a matrix $M$. Then, we should calculate the matrix $C$ such that $CC^{T}=\Sigma$ thanks to a Cholesky decomposition. To do so, the matrix $\Sigma$ has to be definitive positive, which is not always the case. \\

\begin{tcolorbox}[breakable,colback=cyan,opacityfill=0.05,colframe=blue,width=\dimexpr\textwidth+12mm\relax,enlarge left by=-6mm]
\begin{center}
\vspace{0.2cm}
\textbf{How to make a matrix $\Sigma$ definite positive ?}
\end{center}
\begin{enumerate}
    \item Diagonalize the matrix $\Sigma$, \textit{ie} find $P$ an inversible matrix and $D$ a diagonal matrix such that $\Sigma=PDP^{-1}$
    \item Let $(d_{ii})_{i \in \{1,\ldots,d\}}$ be the diagonal elements of the matrix $D$. We calculate : $$\forall i=1,\ldots,d, d_{ii}^{*}=\begin{cases}
        d_{ii}& \text{ if } d_{ii}>0 \\
        10^{-12}& \text{ if } d_{ii}\leq0
    \end{cases}$$
 \item Writing $D^{*}$ the diagonal matrix with $d_{ii}^{*}$ the diagonal elements, we can compute $$\Sigma^{*}=PD^{*}P^{-1}$$ where $\Sigma^{*}$ is the corrected definite positive matrix.
\end{enumerate}
\end{tcolorbox}
Once the matrix $C$ is computed, we can calculate the matrix $Z=CM$ which gives a realization of a multivariate gaussian distribution whose correlation matrix is $\Sigma$.
\subsection{Brownian bridge and final payoff computation}
\label{subsec:brownian-bridge}
The problem of the discretization of the forward equation is that we lose the continuous monitoring of the BRC's barrier. In order to catch the continuity of the barrier without using a very high number of steps in the time grid, we decided to use brownian bridges.\\

Let's first reason with only one underlying $(S_{t})_{t\geq0}$ of volatility $\sigma$. The probability to reach down the barrier $B$ between $t_{k}$ and $t_{k+1}$, knowing that $S_{t_{k}}=x$ and $S_{t_{k+1}}=y$, is given by \footnote{See Appendix \ref{appendix:brownian-bridge} for the demonstration of this formula}:
$$p(x,y,T,B,\sigma) = 
\begin{cases}
1 &\text{ if } x\leq B \text{ or } y\leq B \\
\exp(-2\frac{\log(\frac{x}{B})\log(\frac{y}{B})}{\sigma^{2}T}) &\text{ otherwise }
\end{cases}$$
The probability that the underlying has not touched the barrier during the whole lifetime of the asset is then $p=\prod_{i=1}^{m}(1-p_{i})$ where $p_{i}=p(S_{t_{i}},S_{t_{i+1}},\Delta_{i},B,\sigma_{i})$. When computing the payoff, one has to take into account this probability as follows :
$$\Phi(S) = cap -(K-S_{T})_{+}(1-p)$$

When it comes to several underlyings, the computation of $p$ becomes harder. As explained in \ref{sec:multi-BRC-definition}, the barrier can be a \textit{WorstOf} or a \textit{BestOf}. This means we can have 4 cases :
\begin{enumerate}
    \item \textbf{The barrier is a \textit{WorstOf} and is down} : the barrier is touched if at least one of the underlying has touched the barrier, which is the complementary of the event "Not any underlying has touched the barrier". We then have $p_{i}=1-\prod_{j=1}^{d}(1-p_{i}^{j})$, where $p_{i}^{j}$ is the probability that the $j^{th}$ underlying as reached the barrier between $t_{i}$ and $t_{i+1}$.
    
    \item \textbf{The barrier is a \textit{WorstOf} and is up} : the barrier is touched if all the underlyings have, thus $p_{i}=\prod_{j=1}^{d}p_{i}^{j}$
    
    \item \textbf{The barrier is a \textit{BestOf} and is down} : this situation is the same as above, so we have $p_{i}=\prod_{j=1}^{d}p_{i}^{j}$
    
    \item \textbf{The barrier is a \textit{BestOf} and is up} : this situation is the same as the first one so  $p_{i}=1-\prod_{j=1}^{d}(1-p_{i}^{j})$
\end{enumerate}
Note that in order to compute those probabilities, we have made the strong assumption of independence between the different underlyings, which is not really the case. Nevertheless, this approximation is necessary to have simple formulas and take them into account in the payoff computation.
\section{Resolution of the backward equation}
\subsection{Choice of the function f}
\label{subsec:choice-of-f}
In order to find the expression of $f$ and to give more intuition about what the processus $(Z_{t})_{t\geq0}$, we set up a self-financing portfolio $Y_{t}$ and buy $a_{t}$ stocks $S_{t}$ at time $t$. The rest of the portfolio is invested in a bond whose risk-free rate is $r$. The value of the portfolio should then evolve like this :
$$dY_{t} = r(Y_{t}-a_{t}S_{t})\text{d}t + a_{t}dS_{t}$$
$$dY_{t} = r(Y_{t}-a_{t}S_{t})\text{d}t + a_{t}(rS_{t}\text{d}t+\sigma S_{t}\text{d}W_{t})$$
We can rewrite this :
$$dY_{t} = rY_{t}\text{d}t + \sigma a_{t}S_{t}\text{d}W_{t}$$
The processus $(Y_{t})_{t\geq0}$ presented in \ref{sec:SDE-presentation} is a solution of this SDE with $f(t,Y_{t})=-rY_{t}$, $Z_{t} = \sigma a_{t}S_{t}$ and the terminal condition $\Phi(S)=Y_{T}$.
The hedging-strategy corresponds to $Z_{t}=\sigma S_{t} a_{t}$ where $a_{t}$ is the $\Delta$ of the option. 
\subsection{Resolution formula}
Let us rewrite the backward equation at $t_{k}$ and $t_{k+1}$, two consecutive times in the time grid :
$$Y_{t_{k}} = \Phi(S) + \int_{t_{k}}^{T} f(s,Y_{s}) ds - \int_{t_{k}}^{T} Z_{s} dW_{s}$$
$$Y_{t_{k+1}} = \Phi(S) + \int_{t_{k+1}}^{T} f(s,Y_{s}) ds - \int_{t_{k+1}}^{T} Z_{s} dW_{s}$$
Now let's make the difference between both :
\begin{equation}
    Y_{t_{k}} = Y_{t_{k+1}} + \int_{t_{k}}^{t_{k+1}} f(s,Y_{s}) ds - \int_{t_{k}}^{t_{k+1}}Z_{s} dW_{s}
    \label{eq:Y_t}
\end{equation}
Let $(\mathcal{F}_{t})_{t\geq0}$ be a filtration so that $(Y_{t})_{t\geq0}$ and $(S_{t})_{t\geq0}$ are adapted, and let's apply $\mathbb{E}(.|\mathcal{F}_{t_{k}})$ to the previous equation :
$$\mathbb{E}(Y_{t_{k}}|\mathcal{F}_{t_{k}}) = \mathbb{E}(Y_{t_{k+1}}|\mathcal{F}_{t_{k}}) + \mathbb{E}\left(\int_{t_{k}}^{t_{k+1}} f(s,Y_{s}) ds~\bigg\vert~\mathcal{F}_{t_{k}}\right) - \mathbb{E}\left(\int_{t_{k}}^{t_{k+1}}Z_{s} dW_{s}~\bigg\vert~\mathcal{F}_{t_{k}}\right)$$

The last term is equal to zero because $\mathbb{E}(\int_{t_{k}}^{t_{k+1}}Z_{s} dW_{s}|\mathcal{F}_{t_{k}})=\mathbb{E}(\int_{t_{k}}^{t_{k+1}}Z_{s} dW_{s})$ thanks to the independence, and $\mathbb{E}(\int_{t_{k}}^{t_{k+1}}Z_{s} dW_{s})=0$ as being the expectation of an Ito's integral.\\
For the deterministic term, we have $\mathbb{E}(\int_{t_{k}}^{t_{k+1}}f(s,Y_{s}) ds|\mathcal{F}_{t_{k}})=\mathbb{E}(\int_{t_{k}}^{t_{k+1}}f(s,Y_{s}) ds) = \int_{t_{k}}^{t_{k+1}}f(s,Y_{s}) ds$.
We thus have :
\begin{equation}
    Y_t = \mathbb{E}(Y_{t_{k+1}}|\mathcal{F}_{t_{k}}) + \int_{t_{k}}^{t_{k+1}}f(s,Y_{s}) ds
\end{equation}
Let us denote $\Delta_{k}=t_{k+1}-t_{k}$. We will consider the following approximation for the integral :
$$\int_{t_{k}}^{t_{k+1}}f(s,Y_{s})ds \simeq \Delta_{t_{k}}\theta f(t_{k},Y_{t_{k}}) + \Delta_{t_{k}}(1-\theta) f(t_{k+1},Y_{t_{k+1}})$$
and choose $\theta=1$. The resolution formula becomes :
\begin{equation}
Y_{t_{k}} = \mathbb{E}(Y_{t_{k+1}}|\mathcal{F}_{t_{k}}) + \Delta_{t_{k}}f(t_{k},Y_{t_{k}})
\label{resolutionFormula}
\end{equation}
This has to be calculated for every step in the time grid and for every simulation of the Monte Carlo method.
\subsection{Computation of the conditional expectation}
\label{subsec:conditional-expectation}
The above formula shows that we have to estimate at each step the conditional expectation $\mathbb{E}(Y_{t+1}|\mathcal{F}_{t})$. The kernel regression method has been chosen in order to do so.\\

This method consists in the estimation of $\mathbb{E}(Y|X=x)$ for a given $x$. The estimator proposed by both Nadaraya\cite{nadaraya1964estimating} and Watson\cite{watson1964smooth} in 1964 is the following : $$\hat{m}_{h}(x)=\frac{\sum_{i=1}^{N}K_{h}(x-x_{i})y_{i}}{\sum_{i=1}^{N}K_{h}(x-x_{i})}$$
with $N$ being the number of observations drawn independently.\\

\begin{proof}
The conditional expectation can be written $$\mathbb{E}(Y|X=x)=\int yf(y|x)dy = \int y\frac{f(x,y)}{f(x)}dy$$
We can then use the kernel density estimation for $f(x,y)$ and $f(x)$ with a kernel $K_{h}$ whose bandwidth is $h$ : $$\hat{f}(x,y) = \frac{1}{N}\sum_{i=1}^{N}K_{h}(x-x_{i})K_{h}(y-y_{i})$$
$$\hat{f}(x) = \frac{1}{N}\sum_{i=1}^{N}K_{h}(x-x_{i})$$

We get $$\hat{\mathbb{E}}(Y|X=x) = \int y\frac{\sum_{i=1}^{N}K_{h}(x-x_{i})K_{h}(y-y_{i})}{\sum_{i=1}^{N}K_{h}(x-x_{i})}dy$$
$$=\frac{\sum_{i=1}^{N}K_{h}(x-x_{i})\int y K_{h}(y-y_{i})dy }{\sum_{i=1}^{N}K_{h}(x-x_{i})}$$
$$=\frac{\sum_{i=1}^{N}K_{h}(x-x_{i})y_{i}}{\sum_{i=1}^{N}K_{h}(x-x_{i})}$$
\end{proof}

Note that in our case, the problem is multivariate as the BRC can have several underlyings. The kernel thus becomes : $$K_{H}(x-x_{i}) = |H|^{-\frac{1}{2}}K(H^{-\frac{1}{2}}(x-x_{i}))$$
where $x=(x_{1},\ldots,x_{d})^{T}$, $x_{i}=(x_{1i},\ldots,x_{di})^{T}$, $d$ is the number of underlyings and $H$ is the bandwidth matrix, symmetric and definite positive.\\

Furthermore, we chose to use the standard multivariate gaussian kernel which can be expressed as follows :
$K_{H}(x)=\frac{1}{(2\pi)^{\frac{d}{2}}}|H|^{-\frac{1}{2}}e^{-\frac{1}{2}x^{T}H^{-1}x}$

The choice of matrix $H$ appears to be crucial as it controls the smoothing of the estimation. A good choice would be the \textit{rule of thumb} proposed by Silverman\cite{silverman1986density} which consists in a diagonal matrix whose terms are : $$\forall i=1,\ldots,d, H_{ii} = (\frac{4}{d+2})^{\frac{2}{d+4}}N^{\frac{-2}{d+4}}\sigma_{i}^{2}$$
with $\sigma_{i}^{2}$ the variance of the $d^{th}$ variable.\\

The estimator then finally becomes :
$$\hat{m}_{H}(x)= \frac{\sum_{i=1}^{N}y_{i}e^{-\frac{1}{2}(\frac{4}{N(d+2)})^{-\frac{2}{d+4}}\sum_{j=1}^{d}\frac{(x_{j}-x_{i})^{2}}{\sigma_{j}^{2}}}}{\sum_{i=1}^{N}e^{-\frac{1}{2}(\frac{4}{N(d+2)})^{-\frac{2}{d+4}}\sum_{j=1}^{d}\frac{(x_{j}-x_{i})^{2}}{\sigma_{j}^{2}}}}$$
\subsection{Picard's method}
We can see from the resolution formula \eqref{resolutionFormula} that the value of $Y_{t_{k}}$ we want to compute appears on the left and right side of the equation. We will thus use a fixed-point argument to compute $Y_{t_{k}}$.

\begin{prop}
The application $\Psi : Y \rightarrow \mathbb{E}(Y_{t_{k+1}}|\mathcal{F}_{t_{k}}) + \Delta_{t}f(t_{k},Y_{t_{k}})$ is a contraction of $\mathcal{L}_{2}(\mathcal{F}_{t_{k}})$ for a small $\Delta_{t_{k}}$.
\end{prop}

\begin{proof}
Let $Y_{1}$ and $Y_{2}$ be two elements of $\mathcal{L}_{2}(\mathcal{F}_{t_{k}})$. We have : $$|\Psi(Y_{2})-\Psi(Y_{1})|=\Delta_{t_{k}}|f(t_{k},Y_{2})-f(t_{k},Y_{1})|$$
$$=\Delta_{t_{k}}r |Y_{2}-Y_{1}|\leq \Delta_{t_{k}} C_{r}|Y_{2}-Y_{1}|$$
where $C_{r}$ is a constant depending on $r$.
\end{proof}
Thanks to this, we can apply Picard iterations to find $Y_{t_{k}}$. The methodology is the following:
\begin{itemize}
    \item $i=0$ (first iteration) : $Y_{t_{k}}=0$
    \item $i>0$ (next iterations) : $Y_{t_{k}}^{i}=\mathbb{E}(Y_{t_{k+1}}|\mathcal{F}_{t}) + \Delta_{t}f(t_{k},Y_{t_{k}}^{i-1})$
\end{itemize}
We stop the iterations when the value seems to have converged, \textit{ie} when the difference between two iterations is less than $10^{-8}$.
\subsection{Computation at time $t=0$}
\label{subsec:computation-0}
At time $t=0$, the resolution formula \eqref{resolutionFormula} becomes $$Y_{t_{0}} = \mathbb{E}(Y_{t_{1}}|\mathcal{F}_{t_{0}}) + \Delta_{t_{0}}f(t_{0},Y_{t_{0}})=\mathbb{E}(Y_{t_{1}}) + \Delta_{t_{0}}f(t_{0},Y_{t_{0}})$$
No more kernel estimation is then needed, the computation of $Y_{t_{0}}$ only requires the computation of an expectation (which is approximated by the mean of the vector $(Y_{1,i})_{i=1,\ldots,N}$) and Picard iterations. We use the estimator :
$$Y_{t_{0}} = \frac{1}{N}\sum_{i=1}^{N}Y_{1,i}-r\Delta_{t_{0}}Y_{t_{0}}$$
\subsection{Computation of the $\Delta$ of the option}
As explained in \ref{subsec:choice-of-f}, the process $(\Delta_t)_{0 \leq t \leq T}$ is given by $\Delta_t=\frac{Z_t}{\sigma S_t}$. In order to compute $\Delta_{t_0}$, we then have to compute $Z_{t_0}$.
Let us rewrite equation \ref{eq:Y_t} multiplied by $\Delta W_{t_k} = W_{t_k+1}-W_{t_k}$ and taken with the conditional expectation $\mathbb{E}(.|F_{t_k})$:
$$\mathbb{E}(\Delta W_{t_k} Y_{t_{k}}|F_{t_k}) = \mathbb{E}(\Delta W_{t_k} Y_{t_{k+1}}|F_{t_k}) + \mathbb{E}\left(\Delta W_{t_k} \int_{t_{k}}^{t_{k+1}} f(s,Y_{s}\right) ds~\bigg\vert~F_{t_k}) - \mathbb{E}\left(\Delta W_{t_k} \int_{t_{k}}^{t_{k+1}}Z_{s} dW_{s}~\bigg\vert~F_{t_k}\right)$$

This can be written :
$$Y_{t_{k}}\mathbb{E}(\Delta W_{t_k}) = \mathbb{E}(\Delta W_{t_k} Y_{t_{k+1}}|F_{t_k}) + \mathbb{E}\left(\Delta W_{t_k} \int_{t_{k}}^{t_{k+1}} f(s,Y_{s}) ds\right)- \mathbb{E}\left(\Delta W_{t_k} \int_{t_{k}}^{t_{k+1}}Z_{s} dW_{s}~\bigg\vert~F_{t_k}\right)$$

We know that :
\begin{itemize}
    \item $\mathbb{E}(\Delta W_{t_k})=0$
    \item $\mathbb{E}(\Delta W_{t_k} \int_{t_{k}}^{t_{k+1}} f(s,Y_{s}) ds)=\int_{t_{k}}^{t_{k+1}} f(s,Y_{s}) ds \mathbb{E}(\Delta W_{t_k})=0$
    \item $\mathbb{E}(\Delta W_{t_k} \int_{t_{k}}^{t_{k+1}}Z_{s} dW_{s}|F_{t_k}) \simeq \mathbb{E}(\Delta W_{t_k}^2 Z_{t_k}|F_{t_k}) = Z_{t_k} \mathbb{E}(\Delta W_{t_k}^2|F_{t_k}) = Z_{t_k} \mathbb{E}(\Delta W_{t_k}^2) = Z_{t_k} \Delta_{t_k}$
\end{itemize}

We can then compute $Z_{t_k}$ for each $t_{k}$ with the following formula :
\begin{equation}
    Z_{t_k} \simeq \frac{1}{\Delta_{t_k}}\mathbb{E}(\Delta W_{t_k} Y_{t_{k+1}}|F_{t_k})
\end{equation}

And $\Delta_{t_0}$ is given by :
$$\Delta_{t_0} \simeq \frac{1}{\sigma \Delta_{t_k} S_{t_k}}\mathbb{E}(\Delta W_{t_0} Y_{t_{1}}|F_{t_0})=\frac{1}{\sigma \Delta_{t_k} S_{t_k}}\mathbb{E}(\Delta W_{t_0} Y_{t_{1}})$$

\section{Resolution of a double reflected backward SDE}
\subsection{Motivations}
There are two aspects of the callable barrier reverse convertible that can not be included in the final payoff. These are the coupons ,that can be paid throughout the whole life of the asset, and the callability of the BRC that can be activated at determined times before maturity. This can introduce some discontinuity in the price of the asset, which we tried to integrate in our modelisation with reflected forward backward stochastic differential equations.
\subsection{Description of the new problem and resolution}
We define the following set of equations :
$$S_{t}=S_{0} + \int_{0}^{t}b(s,S_{s})ds + \int_{0}^{t}\sigma(s,S_{s})dW_{s}$$
$$Y_{t}=\Phi(S_{T})+\int_{t}^{T}f(s,Y_{s},Z_{s})ds+(K_{T}^{+}-K_{t}^{+})+(K_{T}^{-}-K_{t}^{-})-\int_{t}^{T}Z_{s}dW_{s}$$
$$\forall t\leq T, L_{t}\leq Y_{t}\leq U_{t} \text{ with }$$ $$\int_{0}^{T}(Y_{s}-L_{s})dK_{s}^{+}=\int_{0}^{T}(U_{s}-Y_{s})dK_{s}^{+}=0$$
as a double reflected forward backward stochastic differential equation (double RFBSDE).

The processes $(L_{t})_{0\leq t \leq T}$ and $(U_{t})_{0\leq t \leq T}$ are called the obstacles and the processes $(K_{t}^{+})_{0\leq t \leq T}$ and $(K_{t}^{-})_{0\leq t \leq T}$ are here to "push" the price respectively above and below those obstacles. \\

The first step to resolve this new problem is to resolve the unreflected BSDE on the time interval $[t_{k-1},t_{k}]$, just as we described in the previous section. We then compute : $$\widetilde{\widetilde{Y}}_{t_{k-1}}=Y_{t_{k}}+\int_{t_{k-1}}^{t_{k}}f(s,Y_{s},Z_{s})ds-\int_{t_{k-1}}^{t_{k}}Z_{s}dW_{s}$$

Then, if the date $t_{k-1}$ is a coupon payment date, we adjust the price and $\widetilde{Y}_{t_{k-1}}= \widetilde{\widetilde{Y}}_{t_{k-1}} + K_{t_{k-1}}^{+}$ with $K_{t_{k-1}}^{+}$ the value of the coupon rate multiplied by the value of the nominal. If $t_{k-1}$ is not a coupon payment date, then $K_{t_{k-1}}^{+}=0$. \\
Finally, if the date $t_{k-1}$ is a call date, we have to make sure that the price of the asset doesn't go up the value at which it can be bought by the issuer. Again, we have to adjut the price. We define :
$$U_{t_{k-1}}=
\begin{cases}
\text{Value of the call} & \text{if } \widetilde{Y}_{t_{k-1}}\geq \text{Value of the call }\\
0& \text{otherwise }
\end{cases}$$

And $Y_{t_{k-1}}=\min(\widetilde{Y}_{t_{k-1}},U_{t_{k-1}})$. We can define $K_{t_{k-1}}=Y_{t_{k-1}}-\widetilde{Y}_{t_{k-1}}$ so that $(K_{t})_{0\leq t\leq T}$ sticks with its definition.
\section{Taking into account the default risk}
The price of the asset still doesn't include the probability of default of the issuer. In order to do that we have to multiply the price by the survival probability of the issuer.\\

We thus use a reduced-form model, which consists in modeling the conditional law of the random time of default $\tau$. In this model, the conditional default probability is given by :
\begin{equation}
    \mathbb{Q}(\tau<t+\text{d}t|\tau\geq t)=\lambda \text{d}t
    \label{eq:reduced-form}
\end{equation}


where $\lambda$ is the intensity of default.\\

Let $F(t)$ denote the distribution function of default time $\tau$. We have $F(t)=\mathbb{P}(\tau\leq t)$ and if $F$ is differentiable, we can define $f$ as $F(t)=\int_{-\infty}^{t}f(u)\text{d}u$.
We can then define the survival probability $S(t) = 1-F(t) = \int_{t}^{\infty}f(u)\text{d}u$.\\

Now let us rewrite equation \ref{eq:reduced-form} :
$$\lambda = \underset{\text{d}t\to 0}{\lim} \frac{\mathbb{Q}(t\leq \tau<t+\text{d}t)}{\text{d}t\mathbb{Q}(\tau\geq t)} = \underset{\text{d}t\to 0}{\lim} \frac{F(t+\text{d}t)-F(t)}{\text{d}t S(t)}=\frac{f(t)}{S(t)}=\frac{-S'(t)}{S(t)}$$

We then have, $-\ln{S(t)}=\int_{0}^{t}\lambda \text{d}u = \lambda t$, and thus $S(t)=e^{-\lambda t}$.

\section{Summary of the pricing methodology}
\label{sec:summary}
We can summarize the pricing methodology we use for callable barrier reverse convertible as follows :
\begin{enumerate}
    \item \textbf{Diffusion of the underlyings} :
    \begin{enumerate}
        \item Compute the correlation matrix $\Sigma$ (make it definite positive if necessary)
        \item Generate a sample of a multivariate normal distribution of dimension $N\times d$ ($N$ is the number of simulations and $d$ the number of underlyings) and covariance matrix  $\Sigma$ 
        \item Diffuse the underlyings with the scheme described in \ref{subsec:underlying-discretization}
    \end{enumerate}
    \item \textbf{Compute the payoff of the BRC} taking into account brownian bridges
    \item \textbf{Resolution of the backward stochastic differential equation} for each step $t_{k}$ of the time grid :
    \begin{enumerate}
        \item Compute the conditional expection $\mathbb{E}(Y_{t_{k+1}}|\mathcal{F}_{t})$ for each simulation as described in \ref{subsec:conditional-expectation}
        \item Compute $\widetilde{\widetilde{Y}}_{t_{k}}=\mathbb{E}(Y_{t_{k+1}}|\mathcal{F}_{t}) -r\Delta_{k}\widetilde{\widetilde{Y}}_{t_{k}}$ using Picard iterations for each simulation
        \item Adjust the price with coupons and compute $\widetilde{Y}_{t_{k}} = \widetilde{\widetilde{Y}}_{t_{k}} + K_{t_{k}}^{+}$
        \item Adjust the price with callability and compute $Y_{t_{k}} = \widetilde{Y}_{t_{k}} + K_{t_{k}}^{-}$
    \end{enumerate}
    \item \textbf{Compute the price at $t=0$} as described in \ref{subsec:computation-0}
    \item \textbf{Take into account the default risk}, \textit{ie} multiply the price by the survival probability of the issuer
\end{enumerate}

%--------------------------------------------------------------------------
%	Chapter 3
%--------------------------------------------------------------------------

\chapter{Results : from a barrier option to a callable BRC}

\section{Results on a simple down-in barrier}
As explained in \ref{sec:BRC-definition}, one of the principal components of a \textit{Barrier Reverse Convertible} is a down and in barrier option on a put. The first step of our approach is thus to price this component with a backward stochastic differential equation resolution. The methodology is the same as described in \ref{sec:summary} without the final adjustments made for coupons and callability, and without the credit default risk.\\

We chose here to work with only one underlying, so we have an analytical formula that gives the theoretical price of the option. Let us take the same notations as in \ref{subsec:brownian-bridge}. For a barrier level $B$ that is less than or equal to the strike $K$, the price $P$ of a down-in put is then given by \cite{hull2016options}: 
$$\begin{aligned}
P =& -S_{0}\mathcal{N}(-x_{1})+Ke^{-rT}\mathcal{N}(-x_{1}+\sigma T)+S_{0}(\frac{B}{S_{0}})^{2\lambda}(\mathcal{N}(y)-\mathcal{N}(y_{1})) \\
&-Ke^{-rT}(\frac{B}{S_{0}})^{2\lambda-2}(\mathcal{N}(y-\sigma T)-\mathcal{N}(y_{1}-\sigma T))
\end{aligned}$$

where :
\begin{itemize}
    \item $\lambda=\frac{r+\sigma^{2}/2}{\sigma^{2}}$
    \item $y=\frac{\ln(\frac{B^{2}}{S_{0}K})}{\sigma \sqrt{T}}+\lambda \sigma \sqrt{T}$
    \item $x_{1}=\frac{\ln(S_{0}/H)}{\sigma \sqrt{T}}+\lambda \sigma \sqrt{T}$
    \item $y_{1}=\frac{\ln(H/S_{0})}{\sigma \sqrt{T}}+\lambda \sigma \sqrt{T}$
    \item $\mathcal{N}(.)$ is the cumulative distribution function of a standard normal variable
\end{itemize}

The different down-in barrier options priced with our methodology are described in the following table :

\begin{table}[H]
\begin{center}
\begin{tabular}{l c c c c c c c} 
\cmidrule(l){1-8} 
Case & $S_{0}$ & K & B & $$\sigma$$ & T & r & Analytical price\\ % Column names row
\midrule % In-table horizontal line
Case 1 	&   100 &   100 &   80  & 	10\%    &   1    &   3\%    & -0.271\\
Case 2	&   100 &   100 &	80  & 	20\%    &   1    &   3\%    & -4.69\\
Case 3 &    100 &   100 &   80  &	50\%    &   1    &   3\%    & -17.788\\
Case 4 	&   100 &   100 &   99  &	10\%    &   1    &   3\%    & -2.626\\
\bottomrule % Bottom horizontal line
\end{tabular}
\end{center}
    \caption{Description of the down-in barrier options used for pricing}
    \label{down-in-barriers}
\end{table}

This setup will allow us to analyse the behaviour and the convergence of our methodology for different cases, and more especially for corner cases like Case 4 (indeed, as the barrier is extremely close to the strike and the volatility high enough, the pricing of this barrier option is actually the pricing of a put option with the same strike, volatility and maturity).

Figure \ref{fig:down-in-barrier-pricing} and Table \ref{tab:barrier-option-pricing} present the results obtained on 100 pricings.

\begin{figure}[H]
\begin{center}
\begin{tikzpicture}[yscale=0.9][xscale=0.5]
\begin{axis}[
    title={Down-in barrier option pricing},
    xlabel={Number of simulations},
    ylabel={Relative error (\%)},
    xmin=0, xmax=10000,
    ymin=-6, ymax=1,
    xtick={2500,5000,7500,10000},
    ytick={-6,-5,-4,-3,-2,-1,0,1},
    legend pos=south east,
    ymajorgrids=true,
    grid style=dashed,
]
\addplot[
    color=violet,
    mark=o,
    ]
    coordinates {
    (1000,-3.57)(2500,-3.53)(5000,-3.10)(7500,-2.58)(10000,-2.59)
    };
\addplot[
    color=blue,
    mark=o,
    ]
    coordinates {
    (1000,-0.98)(2500,-0.85)(5000,-0.74)(7500,-0.53)(10000,-0.54)
    };
\addplot[
    color=green,
    mark=o,
    ]
    coordinates {
    (1000,-0.46)(2500,-0.33)(5000,-0.30)(7500,-0.27)(10000,-0.31)
    };
\addplot[
    color=cyan,
    mark=o,
    ]
    coordinates {
    (1000,-0.67)(2500,-0.55)(5000,-0.55)(7500,-0.44)(10000,-0.47)
    };
\legend{Case 1, Case 2, Case 3, Case 4}
\end{axis}
\end{tikzpicture}
    \label{fig:down-in-barrier-pricing}
    \caption{Results of the down-in barrier pricing}
\end{center}
\end{figure}

\begin{table}
\centering
\begin{tabular}{l c c c c} 
& \multicolumn{4}{c}{Case} \\ 
\cmidrule(l){2-5} 
Nb of simulations & 1 & 2 & 3 & 4\\ % Column names row
\midrule % In-table horizontal line
1000 & -3.57\% & -0.98\% & -0.46\% & -0.67\%\\ % Content row 1
2500 & -3.53\% & -0.85\% & -0.33\% & -0.55\%\\ % Content row 2
5000 & -3.10\% & -0.74\% & -0.30\% & -0.55\%\\ % Content row 3
7500 & -2.58\% & -0.53\% & -0.27\% & -0.44\%\\ % Content row 4
10000 & -2.59\% & -0.54\% & -0.31\% & -0.47\%\\ % Content row 5
\bottomrule % Bottom horizontal line
\end{tabular}
\caption{Relative errors (\%) on the down-in barrier pricing (mean on 100 pricings)}
\label{tab:barrier-option-pricing}
\end{table}

We can see that the convergence is obtained in each case, with an absolute relative error inferior to 3\% in Case 1 and inferior to 1\% for the other cases. Results for the Case 1 seem to be less precise than the others ; this could be explained by the fact that in this case, the barrier is rarely touched and there is only a few simulations which end to a non-worthless payoff. The estimation of the conditional expectation can thus become more difficult and unprecise.

\section{Results on callable BRCs}
After having priced the down-in barrier successfully, we can now price a more complex instrument such as the \textit{Barrier Reverse Convertible}. In order to do that, the whole methodology described in \ref{chap:pricing-methodology} is now applied, that is to say that the coupons and the callability, as well as the credit risk, are now taken into account. \\

For this pricing, we chose three real-world \textit{Barrier Reverse Convertibles}, all uni-dimensional (\textit{i.e.} only one underlying is monitoring the barrier). Their characteristics are gathered in Table \ref{tab:uni-BRC-charachteristics} :

\begin{table}[H]
\centering
\begin{tabular}{l c c c c c c } 
& \multicolumn{6}{c}{Characteristics} \\ 
\cmidrule(l){2-7} 
BRC & Underlying & Volatility & Strike\tablefootnote{This is also the initial level of the underlying} & Barrier Level & Coupon Rate & Maturity\\ % Column names row
\midrule % In-table horizontal line
1 & Swatch & 22.25\% & 474.60 & 355.95 & 1.35\% & 2 years\\ % Content row 1
2 & UBS & 23.11\% & 16.31 & 12.23 & 1.65\% & 2 years\\ % Content row 2
3 & Swisscom & 17.168\% & 444.30 & 399.87 & 1.5\% & 1 year\\ % Content row 3
\bottomrule % Bottom horizontal line
\end{tabular}
\caption{Description of the \textit{Barrier Reverse Convertibles} priced}
\label{tab:uni-BRC-charachteristics}
\end{table}

Graphs \ref{fig:brc-swatch}, \ref{brc-ubs} and \ref{brc-swisscom} sum up the results obtained :

\begin{figure}[H]
\begin{center}
\begin{tikzpicture}[yscale=0.9][xscale=0.8]
\begin{axis}[
    title={BRC on Swatch},
    xlabel={Number of simulations},
    ylabel={Relative error (\%)},
    xmin=0, xmax=20000,
    ymin=97.25, ymax=98.50,
    xtick={5000,10000,15000,20000},
    ytick={97.25,97.50,97.75,98.0,98.25,98.50},
    legend pos=south east,
    ymajorgrids=true,
    grid style=dashed,
]
\addplot[
    color=blue,
    ]
    coordinates {
    (0,97.31)(20000,97.31)
    };
\addplot[
    color=red,
    ]
    coordinates {
    (0,98.31)(20000,98.31)
    };
\addplot[
    color=cyan,
    mark=o,
    ]
    coordinates {
    (2500,97.98387)(5000,98.15167)(7500,98.21007)(10000,98.24800)(20000,97.596942)
    };
    							
\legend{Bid, Ask, Price}

\end{axis}
\end{tikzpicture}
\end{center}
    \label{fig:brc-swatch}
    \caption{Results of the pricing for a BRC on Swatch}
\end{figure}


\begin{figure}[H]
\begin{center}
\begin{tikzpicture}[yscale=0.9][xscale=0.8]
\begin{axis}[
    title={BRC on UBS},
    xlabel={Number of simulations},
    ylabel={Relative error (\%)},
    xmin=0, xmax=20000,
    ymin=96.25, ymax=97.75,
    xtick={5000,10000,15000,20000},
    ytick={96.25,96.50,96.75,97.0,97.25,97.50,97.75,98.0},
    legend pos=south east,
    ymajorgrids=true,
    grid style=dashed,
]
\addplot[
    color=blue,
    ]
    coordinates {
    (0,96.37)(20000,96.37)
    };
\addplot[
    color=red,
    ]
    coordinates {
    (0,97.37)(20000,97.37)
    };
\addplot[
    color=cyan,
    mark=o,
    ]
    coordinates {
    (1000,97.4725301462994)(2500,97.51114)(5000,97.50031)(7500,97.49034)(10000,97.49501)(20000,97.483500)
    };
  							  							
\legend{Bid, Ask, Price}

\end{axis}
\end{tikzpicture}
\label{brc-ubs}
    \caption{Results of the pricing for a BRC on UBS}
\end{center}
\end{figure}


\begin{figure}[H]
\begin{center}
\begin{tikzpicture}[yscale=0.9][xscale=0.8]
\begin{axis}[
    title={BRC on Swisscom},
    xlabel={Number of simulations},
    ylabel={Relative error (\%)},
    xmin=0, xmax=20000,
    ymin=98.25, ymax=100,
    xtick={5000,10000,15000,20000},
    ytick={98.25,98.50,98.75,99.0,99.25,99.50,99.75,100.0},
    legend pos=south east,
    ymajorgrids=true,
    grid style=dashed,
]
\addplot[
    color=blue,
    ]
    coordinates {
    (0,98.68)(20000,98.68)
    };
\addplot[
    color=red,
    ]
    coordinates {
    (0,99.68)(20000,99.68)
    };
\addplot[
    color=cyan,
    mark=o,
    ]
    coordinates {
    (2500,98.79471)(5000,98.91070)(7500,98.94420)(10000,98.95531)(20000,98.978825)
    };
\legend{Bid, Ask, Price}
				
\end{axis}
\end{tikzpicture}
\label{brc-swisscom}
    \caption{Results of the pricing for a BRC on Swisscom}
\end{center}
\end{figure}

\section{Results on multi callable BRCs}
We now apply the pricing methodology on multi-callable \textit{Barrier Reverse Convertibles}. We chose three assets that are described in Table \ref{tab:multi-BRC-charachteristics} :

\begin{table}[H]
\centering
\begin{tabular}{l c c c c c c } 
& \multicolumn{6}{c}{Characteristics} \\ 
\cmidrule(l){2-7} 
BRC & Underlyings & Volatility & Strike\tablefootnote{This is also the initial level of the underlying} & Barrier Level & Coupon Rate & Maturity\\ % Column names row
\midrule % In-table horizontal line
 & SMI & 14.42\% & 8940.46 & 5364.276 & & \\ % Content row 1
1 & EuroStoxx50 & 13.21\% & 3573.76 & 2144.256 & 2.0\% & 3 years\\ % Content row 1
 & Nikkei225 & 16.50\% & 22930.36 & 13758.216 &  &\\ % Content row 1
 \midrule % In-table horizontal line
 & SMI & 14.42\% & 8824.56 & 4324.03 &  & \\ % Content row 1
2 & EuroStoxx50 & 13.21\% & 3269.63 & 1602.12 & 3.24\% & 1 year\\ % Content row 1
 & S\&P500 & 15.23\% & 1972.18 & 966.37 & & \\ % Content row 1
\midrule % In-table horizontal line
 & SMI & 14.42\% & 8949.86 & 5369.916 &  & \\% Content row 1
3 & EuroStoxx50 & 13.21\% & 3433.54 & 2060.124 & 1.8\% & 2 years\\ % Content row 1
 & S\&P500 & 15.23\% & 2438.21 & 1462.926 & &  \\% Content row 1
  & Nikkei225 & 16.50\% & 19537.10 & 11722.260 & & \\ % Content row 1
\bottomrule % Bottom horizontal line
\end{tabular}
\caption{Description of the multi-\textit{Barrier Reverse Convertibles} priced}
\label{tab:multi-BRC-charachteristics}
\end{table}

\begin{figure}[H]
\begin{center}
\begin{tikzpicture}[yscale=0.9][xscale=0.8]
\begin{axis}[
    title={BRC on SMI, EuroStoxx50 and Nikkei225},
    xlabel={Number of simulations},
    ylabel={Relative error (\%)},
    xmin=0, xmax=35000,
    ymin=97, ymax=99,
    xtick={5000,10000,15000,20000,25000,30000,35000},
    ytick={97.0,97.25,97.5,97.75,98,98.25,98.5,98.75,99},
    legend pos=south east,
    ymajorgrids=true,
    grid style=dashed,
]
\addplot[
    color=blue,
    ]
    coordinates {
    (0,97.12)(35000,97.12)
    };
\addplot[
    color=red,
    ]
    coordinates {
    (0,97.9)(35000,97.9)
    };
\addplot[
    color=cyan,
    mark=o,
    ]
    coordinates {
    (2500,98.118750)(5000,98.222870)(7500,98.350130)(10000,98.425353)(20000,98.478750)(25000,98.45757)(30000,98.40819)(35000,98.44820)
    };
\legend{Bid, Ask, Price}
\end{axis}
\end{tikzpicture}
    \label{brc-smi-eurostoxx-nikkei}
    \caption{Results of the pricing for a multi BRC on SMI, EuroStoxx50, Nikkei225}
\end{center}
\end{figure}


\begin{figure}[H]
\begin{center}
\begin{tikzpicture}[yscale=0.9][xscale=0.8]
\begin{axis}[
    title={BRC on SMI, EuroStoxx50 and S\&P500},
    xlabel={Number of simulations},
    ylabel={Relative error (\%)},
    xmin=0, xmax=30000,
    ymin=101.5, ymax=103.5,
    xtick={5000,10000,15000,20000,25000,30000,30000},
    ytick={101.5,101.75,102,102.25,102.5,102.75,103,103.25,103.5},
    legend pos=south east,
    ymajorgrids=true,
    grid style=dashed,
]
\addplot[
    color=blue,
    ]
    coordinates {
    (0,101.61)(35000,101.61)
    };
\addplot[
    color=red,
    ]
    coordinates {
    (0,102.11)(35000,102.11)
    };
\addplot[
    color=cyan,
    mark=o,
    ]
    coordinates {
    (2500,103.1411028)(5000,103.1517035)(7500,103.1552361)(10000,103.1570022)(20000,103.15965)(25000,103.16018)(30000,103.16053)
    };
    						
\legend{Bid, Ask, Price}
\end{axis}
\end{tikzpicture}
    \label{brc-smi-eurostoxx-sp500}
    \caption{Results of the pricing for a multi BRC on SMI, EuroStoxx50, S\&P500}
\end{center}
\end{figure}


\begin{figure}[H]
\begin{center}
\begin{tikzpicture}[yscale=0.9][xscale=0.8]
\begin{axis}[
    title={BRC on SMI, EuroStoxx50, S\&P500 and Nikkei225},
    xlabel={Number of simulations},
    ylabel={Relative error (\%)},
    xmin=0, xmax=40000,
    ymin=98.5, ymax=99.75,
    xtick={5000,10000,15000,20000,25000,30000,35000,40000},
    ytick={98.5,98.75,99,99.25,99.5,99.75},
    legend pos=south east,
    ymajorgrids=true,
    grid style=dashed,
]
\addplot[
    color=blue,
    ]
    coordinates {
    (0,98.57)(40000,98.57)
    };
\addplot[
    color=red,
    ]
    coordinates {
    (0,99.43)(40000,99.43)
    };
\addplot[
    color=cyan,
    mark=o,
    ]
    coordinates {
    (2500,99.59246987)(5000,99.60809215)(7500,99.58864458)(10000,99.61014605)(20000,99.683560)(25000,99.679430)(30000,99.701340)(35000,99.722940)(40000,99.710480)
    };
\legend{Bid, Ask, Price}
\end{axis}
\end{tikzpicture}
    \label{brc-smi-eurostoxx-sp500-nikkei}
    \caption{Results of the pricing for a multi BRC on SMI, EuroStoxx50, S\&P500, Nikkei225}
\end{center}
\end{figure}

\section{Discussion on the results}

\section{Improvement of the methodology with a volatility structure}
\subsection{The need of a volatility structure}
The previous methodology presents very strong and not realistic hypothesis ; for example, the choice of a model whose volatility remains constant over the whole period of pricing is not realistic. In order to improve our methodology and obtain better results, we chose to modelize the volatility with a term structure and interpolate it when required. This presents the advantage not to completely upset our methodology and implementation but still consider a non-constant volatility. It is also convenient since it doesn't need any calibration and since the data of the volatility term structure can be easily found.

\subsection{Gregory-Delbourgo interpolation}
This rational quadratic spline interpolation has been presented by Gregory \& Delbourgo in \cite{gregory1982piecewise}. The main advantage of such an interpolation is that it provides an interpolant of continuity class $C^1$ and that it preserves the shape of the initial data set. Note that this interpolation only works for a strictly monotonic set of points $(x_i,f_i)$, for $i=1,\ldots,n$. \\

In this case, it will be assumed that $f_1 < f_2 < \ldots< f_n$ since the case of strictly decreasing set of points can be treated the same way. The aim of the interpolation is to build a piecewise rational quadratic function $s(x)$ on $[x_1;x_n]$ such that :
$$\forall i=1,\ldots,n, s(x_i)=f_i$$

Let us write $h_i=x_{i+1}-x_i$, $\theta=\frac{x-x_i}{h_i}$ and $\Delta_i=\frac{f_{i+1}-f_i}{h_i}$

We can define, for $i=2,\ldots,n-1$: $$d_i = \begin{cases}
0 & \text{if } f_{i+1}=f_i \\
\frac{\Delta_i\Delta_{i-1}}{(f_{i+1}-f_{i-1})(x_{i+1}-x_{i-1})} & \text{otherwise }
\end{cases}$$

And 
$$d_1 = \begin{cases}
0 & \text{if } f_{1}=f_2 \\
\frac{\Delta_1^2}{(f_{3}-f_{1})(x_{3}-x_{1})} & \text{otherwise }
\end{cases}$$

$$d_n = \begin{cases}
0 & \text{if } f_{n-1}=f_n \\
\frac{\Delta_{n-1}^2}{(f_{n}-f_{n-2})(x_{n}-x_{n-2})} & \text{otherwise }
\end{cases}$$

We then have, for $x\in [x_i;x_{i+1}]$ : $$s(x) = f_i + \frac{(f_{i+1}-f_{i})(\Delta_i\theta^2+d_i\theta(1-\theta))}{\Delta_i+(d_{i+1}+d_i-2\Delta_i)\theta(1-\theta)}$$

This can be rewrite like this :
$$s(x)=\frac{f_{i+1}\theta^2 + \Delta_i^{-1}(f_{i+1}d_i+f_id_{i+1})\theta(1-\theta)+f_i(1-\theta)^2}{\theta^2+\Delta_i^{-1}(d_{i+1}+d_i)\theta(1-\theta)+(1-\theta)^2}$$

For $0 \leq \theta \leq 1$, the denominator is always positive. Furthermore, we can differentiate $s$ for $x\in[x_i;x_{i+1}]$ :
$$s^{(1)}(x) = \frac{d_{i+1}\theta^2+2\Delta_i\theta(1-\theta+d_i(1-\theta)^2)}{(\theta^2+\Delta_i^{-1}(d_{i+1}+d_i)\theta(1-\theta)+(1-\theta)^2)^2}$$

We can then conclude that $\forall x\in[x_i;x_{i+1}]$, $s^{(1)}(x)$ exists and $s^{(1)}(x)>0$, which was expected.

\subsection{Results}

\begin{figure}[H]
\begin{center}
\begin{tikzpicture}[yscale=0.9][xscale=0.8]
\begin{axis}[
    title={BRC on Swatch},
    xlabel={Number of simulations},
    ylabel={Relative error (\%)},
    xmin=0, xmax=20000,
    ymin=97.25, ymax=98.50,
    xtick={5000,10000,15000,20000},
    ytick={97.25,97.50,97.75,98.0,98.25,98.50},
    legend pos=north east,
    ymajorgrids=true,
    grid style=dashed,
]
\addplot[
    color=blue,
    ]
    coordinates {
    (0,97.31)(20000,97.31)
    };
\addplot[
    color=red,
    ]
    coordinates {
    (0,98.31)(20000,98.31)
    };
\addplot[
    color=cyan,
    mark=o,
    ]
    coordinates {
    (2500,97.30292)(5000,97.46841)(7500,97.52690)(10000,97.56498)(20000,97.599087)
    };
  				  							
\legend{Bid, Ask, Price}

\end{axis}
\end{tikzpicture}
\label{brc-swatch-delbourgo}
    \caption{Results of the pricing for a BRC on Swatch}
\end{center}
\end{figure}


\begin{figure}[H]
\begin{center}
\begin{tikzpicture}[yscale=0.9][xscale=0.8]
\begin{axis}[
    title={BRC on UBS},
    xlabel={Number of simulations},
    ylabel={Relative error (\%)},
    xmin=0, xmax=20000,
    ymin=96.25, ymax=99,
    xtick={5000,10000,15000,20000},
    ytick={96.50,97.0,97.50,98.0,98.5,99},
    legend pos=south east,
    ymajorgrids=true,
    grid style=dashed,
]
\addplot[
    color=blue,
    ]
    coordinates {
    (0,96.37)(20000,96.37)
    };
\addplot[
    color=red,
    ]
    coordinates {
    (0,97.37)(20000,97.37)
    };
\addplot[
    color=cyan,
    mark=o,
    ]
    coordinates {
    (1000,98.6587530478919)(2500,98.68647)(5000,98.67670)(7500,98.66755)(10000,98.67211)(20000,98.660945)
    };
   				 							
\legend{Bid, Ask, Price}

\end{axis}
\end{tikzpicture}
\label{brc-ubs-delbourgo}
    \caption{Results of the pricing for a BRC on UBS}
\end{center}
\end{figure}


\begin{figure}[H]
\begin{center}
\begin{tikzpicture}[yscale=0.9][xscale=0.8]
\begin{axis}[
    title={BRC on Swisscom},
    xlabel={Number of simulations},
    ylabel={Relative error (\%)},
    xmin=0, xmax=20000,
    ymin=98.25, ymax=100,
    xtick={5000,10000,15000,20000},
    ytick={98.25,98.50,98.75,99.0,99.25,99.50,99.75,100.0},
    legend pos=south east,
    ymajorgrids=true,
    grid style=dashed,
]
\addplot[
    color=blue,
    ]
    coordinates {
    (0,98.68)(20000,98.68)
    };
\addplot[
    color=red,
    ]
    coordinates {
    (0,99.68)(20000,99.68)
    };
\addplot[
    color=cyan,
    mark=o,
    ]
    coordinates {
    (2500,99.11909)(5000,99.22453)(7500,99.25563)(10000,99.26520)(20000,99.285241)
    };
\legend{Bid, Ask, Price}
				
\end{axis}
\end{tikzpicture}
\label{brc-swisscom-delbourgo}
    \caption{Results of the pricing for a BRC on Swisscom}
\end{center}
\end{figure}

\begin{figure}[H]
\begin{center}
\begin{tikzpicture}[yscale=0.9][xscale=0.8]
\begin{axis}[
    title={BRC on SMI, EuroStoxx50 and Nikkei225},
    xlabel={Number of simulations},
    ylabel={Relative error (\%)},
    xmin=0, xmax=35000,
    ymin=97, ymax=99,
    xtick={5000,10000,15000,20000,25000,30000,35000},
    ytick={97.0,97.25,97.5,97.75,98,98.25,98.5,98.75,99},
    legend pos=south east,
    ymajorgrids=true,
    grid style=dashed,
]
\addplot[
    color=blue,
    ]
    coordinates {
    (0,97.12)(35000,97.12)
    };
\addplot[
    color=red,
    ]
    coordinates {
    (0,97.9)(35000,97.9)
    };
\addplot[
    color=cyan,
    mark=o,
    ]
    coordinates {
    (2500,98.118750)(5000,98.222870)(7500,98.350130)(10000,98.425353)(20000,98.478750)(25000,98.45757)(30000,98.40819)(35000,98.44820)
    };
\legend{Bid, Ask, Price}
\end{axis}
\end{tikzpicture}
    \label{brc-smi-eurostoxx-nikkei-delbourgo}
    \caption{Results of the pricing for a multi BRC on SMI, EuroStoxx50, Nikkei225}
\end{center}
\end{figure}


\begin{figure}[H]
\begin{center}
\begin{tikzpicture}[yscale=0.9][xscale=0.8]
\begin{axis}[
    title={BRC on SMI, EuroStoxx50 and S\&P500},
    xlabel={Number of simulations},
    ylabel={Relative error (\%)},
    xmin=0, xmax=30000,
    ymin=101.5, ymax=103.5,
    xtick={5000,10000,15000,20000,25000,30000,30000},
    ytick={101.5,101.75,102,102.25,102.5,102.75,103,103.25,103.5},
    legend pos=south east,
    ymajorgrids=true,
    grid style=dashed,
]
\addplot[
    color=blue,
    ]
    coordinates {
    (0,101.61)(35000,101.61)
    };
\addplot[
    color=red,
    ]
    coordinates {
    (0,102.11)(35000,102.11)
    };
\addplot[
    color=cyan,
    mark=o,
    ]
    coordinates {
    (2500,103.1411028)(5000,103.1517035)(7500,103.1552361)(10000,103.1570022)(20000,103.15965)(25000,103.16018)(30000,103.16053)
    };
    						
\legend{Bid, Ask, Price}
\end{axis}
\end{tikzpicture}
    \label{brc-smi-eurostoxx-sp500-delbourgo}
    \caption{Results of the pricing for a multi BRC on SMI, EuroStoxx50, S\&P500}
\end{center}
\end{figure}


\begin{figure}[H]
\begin{center}
\begin{tikzpicture}[yscale=0.9][xscale=0.8]
\begin{axis}[
    title={BRC on SMI, EuroStoxx50, S\&P500 and Nikkei225},
    xlabel={Number of simulations},
    ylabel={Relative error (\%)},
    xmin=0, xmax=40000,
    ymin=98.5, ymax=99.75,
    xtick={5000,10000,15000,20000,25000,30000,35000,40000},
    ytick={98.5,98.75,99,99.25,99.5,99.75},
    legend pos=south east,
    ymajorgrids=true,
    grid style=dashed,
]
\addplot[
    color=blue,
    ]
    coordinates {
    (0,98.57)(40000,98.57)
    };
\addplot[
    color=red,
    ]
    coordinates {
    (0,99.43)(40000,99.43)
    };
\addplot[
    color=cyan,
    mark=o,
    ]
    coordinates {
    (2500,99.59246987)(5000,99.60809215)(7500,99.58864458)(10000,99.61014605)(20000,99.683560)(25000,99.679430)(30000,99.701340)(35000,99.722940)(40000,99.710480)
    };
\legend{Bid, Ask, Price}
\end{axis}
\end{tikzpicture}
    \label{brc-smi-eurostoxx-sp500-nikkei-delbourgo}
    \caption{Results of the pricing for a multi BRC on SMI, EuroStoxx50, S\&P500, Nikkei225}
\end{center}
\end{figure}

%--------------------------------------------------------------------------
%	Conclusion
%-----------------------------------------------------------------------
\backmatter
\chapter*{Conclusion}
\addcontentsline{toc}{chapter}{Conclusion}


%%%%%%%%
% BIBLIO
%%%%%%%%
\newpage
\nocite{*}

\bibliographystyle{plain}
\bibliography{bibli}


%--------------------------------------------------------------------------
%	Appendix
%--------------------------------------------------------------------------

\newpage
\begingroup
\let\clearpage\relax
\let\cleardoublepage\relax


\renewcommand{\thesection}{\Alph{section}}
\chapter*{Appendix}
\addcontentsline{toc}{chapter}{Appendix}
\appendix
\fancyhead[R]{Appendix}

\section{Generalities on continuous barrier options}
\label{appendix:down-in-put}

\subsection{Brief description of continuous barrier options}
A continuous barrier option is a path-dependent option whose final payoff depends on whether the price of the underlying asset has reached a certain barrier level or not. The monitoring of the barrier is made continuously during the lifetime of the option. The payoff of those options is generally the one of a call or a put.\\

The four main types of continuous barrier options are the following :
\begin{itemize}
    \item \textbf{Up-and-Out}: The barrier is reached if the price of the underlying happens to be above the barrier level. In this case, the option becomes worthless.
    \item \textbf{Down-and-Out} : The barrier is reached if the price of the underlying happens to be below the barrier level. In this case, the option becomes worthless.
    \item \textbf{Up-and-In} : The barrier is reached if the price of the underlying happens to be above the barrier level. In this case, the option is activated.
    \item \textbf{Down-and-In} : The barrier is reached if the price of the underlying happens to be below the barrier level. In this case, the option is activated.
\end{itemize}

\subsection{In-out parity}
Let us assume one holds two continuous barrier options with the same barrier level. The first one is a X-and-in option $O_{in}$, whereas the second one is a X-and-out option $O_{out}$. Two situations can happen : 
\begin{itemize}
    \item The price of the underlying reaches the barrier during the lifetime of the asset, then the X-and-in option is activated and pays at the end the payoff of the vanilla option $O$.
    \item The price of the underlying does not reache the barrier during the lifetime of the asset, then the X-and-out option pays at the end the payoff of the vanilla option $O$.
\end{itemize}

The final payoff of this basket of options is thus the same as the one of the corresponding vanilla option. By virtue of the no-arbitrage condition, we can write the in-out parity as follows : 
$$O = O_{in} + O_{out}$$

The price of a barrier option is thus less expensive than its vanilla counterpart. Moreover, this formula is valid for continuous barrier options on all european-style assets, and more especially continuous barrier options on calls and puts.

\subsection{Pricing of continuous barrier options}
Barrier options can be defined by their final expected payoff. Let us denote for example $V(S,t)$ the price of a down-and-out option on a call whose barrier level is $B$. We have : 
$$V(S,0) = e^{-rT}\mathbb{E}((S_T-K)_+\mathbf{1}_{\left\{\underset{0 \leq t \leq T}{\min} S_t>B\right\}})$$

The price $V$ also satisfies the Black-Scholes pricing PDE :
\begin{equation}
    \frac{\partial V}{\partial t} + \frac{1}{2}\sigma^2(S_t,t)S_t^2 \frac{\partial^2 V}{\partial S^2} + rS\frac{\partial V}{\partial S}-rV=0
    \label{eq:BS-PDE}
\end{equation}


with some boundary conditions :
\begin{itemize}
    \item For an 'out' barrier option, $V(B,t)=0$ for $t<T$. The Black-Scholes PDE is then solved for $S\in[0;B]$. Furthermore, a terminal condition is required ; for a down-and-out barrier option on a call, it would be for example $V(S,T)=\text{max}(S-K,0)\mathbf{1}_{S>B}$
    \item For an 'in' option, there is no payoff if the barrier is not reached, which gives the condition $V(S,T)=0$. However, if the barrier is triggered, we have $V(B,t)=C(B,t)$, with $C(B,t)$ the price of the corresponding vanilla option.
\end{itemize}

Under the Black-Scholes model (\textit{ie} with a constant interest rate $r$ and a constant volatility $\sigma$), the PDE (\ref{eq:BS-PDE}) can be analytically solved for barrier options. Analytical formulas can be found in \cite{hull2016options} and \cite{wilmott1998sons} ; for example, if the strike $K$ is above the barrier $B$, the price of a down-and-out call option is :
$$V(S,t)=C(S,t)-C(B^2,t)(\frac{S}{B})^{1-\frac{2r}{\sigma}}$$

\section{Demonstration of the brownian bridge formula}
\label{appendix:brownian-bridge}
Let $(S_t)_{0 \leq t \leq T}$ be a normal process, $B$, $t_i$ and $t_{i+1}$ real numbers such that $t_{i+1}-t_i>0$. We have : 
\begin{equation}
    \mathbb{P}(\exists t \in [t_i,t_{i+1}],S_t>B~|~S_{t_i}=s_i,S_{t_{i+1}}=s_{i+1}) = e^{-2 \frac{(s_i-B)(s_{i+1}-B)}{\sigma^2(t_{i+1}-t_i)}}
    \label{eq:brownian-bridge}
\end{equation}
    
\begin{proof}
We know that :
$$\mathcal{L}((S_t)_{t_i \leq t \leq t_{i+1}})=\mathcal{L}((s_i+\sigma W_{t-t_i})_{t-t_i})$$
where $(W_t)_{t_i \leq t \leq t_{i+1}}$ is a brownian motion.\\

We thus have :
$$\begin{aligned}
(\ref{eq:brownian-bridge}) = & \mathbb{P}\left(\exists t \in [t_i,t_{i+1}],s_i+\sigma W_{t-t_i}>B~\bigg\vert~W_{t_{i+1}-t_i}=\frac{s_{i+1}-s_i}{\sigma}\right) \\
= & \mathbb{P}\left(\exists t \in [t_i,t_{i+1}],W_{t-t_i}>\frac{B-s_i}{\sigma}~\bigg\vert~W_{t_{i+1}-t_i}=\frac{s_{i+1}-s_i}{\sigma}\right)\\
= & \mathbb{P}\left(\underset{t \in [t_i,t_{i+1}]}{\max} W_{t-t_i}>\frac{B-s_i}{\sigma}~\bigg\vert~W_{t_{i+1}-t_i}=\frac{s_{i+1}-s_i}{\sigma}\right)
\end{aligned}$$

Let us note $u=t-t_i$ and $T=t_{i+1}-t_{i}$.
$$\begin{aligned}
(\ref{eq:brownian-bridge}) = \mathbb{P}\left(\underset{u \in [0,T]}{\max} W_{u}>\frac{B-s_i}{\sigma}~\bigg\vert~W_{T}=\frac{s_{i+1}-s_i}{\sigma}\right)
\end{aligned}$$

But we know that, for $(W_t)_{t \geq 0}$ a brownian motion, $a,b \in \mathbb{R}$, $c=b-a$, using the reflection principle, we have :
$$\begin{aligned}
\mathbb{P}\left(\underset{0 \leq s \leq T}{\max} W_{s}>a~\bigg\vert~W_{T}=-c\right) = e^{\frac{2ab}{T}}
\end{aligned}$$

Here, $a=\frac{B-s_i}{\sigma}$, $b=\frac{B-s_{i+1}}{\sigma}$, so :
$$\begin{aligned}
\mathbb{P}(\exists t \in [t_i,t_{i+1}],S_t>B~|~S_{t_i}=s_i,S_{t_{i+1}}=s_{i+1}) = e^{\frac{-2(s_i-B)(s_{i+1}-B)}{\sigma^2 T}}
\end{aligned}$$
\end{proof}

Note that in our case, $(S_t)_{0 \leq t \leq T}$ being a log-normal process, this formula becomes :
$$\begin{aligned}
\mathbb{P}(\exists t \in [t_i,t_{i+1}],S_t>B~|~S_{t_i}=s_i,S_{t_{i+1}}=s_{i+1}) = e^{\frac{-2\log{\frac{s_i}{B}}\log{\frac{s_{i+1}}{B}}}{\sigma^2 T}}
\end{aligned}$$

\endgroup

\end{document}



